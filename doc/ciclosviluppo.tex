\chapter{Ciclo di sviluppo}

La metodologia suggerita prevede una fase di analisi monolitica, che produca come artefatti una modellazione dei casi dʼ uso, attività, classi e sequenza e una di realizzazione iterativa ed incrementale ispirata ad una delle varie metodologie agili. In prima fase, quindi, ho sviluppato il Documento di Analisi con i vari diagrammi. In seguito, poi, mi sono dedicato alla progettazione dell'applicazione. Dopo la stesura del primo documento di progettazione mi sono dedicato allo sviluppo della prima release dell'applicazione andando a eliminare i problemi e le incongruenze sorte, risolvendo i problemi ad alto rischio prima e tralasciando quelli banali. Cosi facendo ho avuto modo di consentire un rapido sviluppo di versioni via via più complete dell'applicazione. Una volta testato il lato server (business logic) mediante utilizzo di Junit ho potuto dedicarmi all'aspetto grafico curandone i dettagli. Il tutto è stato frequentemente "committato" su github (\href{https://github.com/jgemmy/Doodle}{Github}) e (\href{https://bitbucket.org/jgemmy/doodle}{Bitbitbucket}).
